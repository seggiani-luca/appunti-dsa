\documentclass[a4paper,12pt]{article}

\usepackage[french,italian]{babel}
\usepackage[T1]{fontenc}
\usepackage[utf8]{inputenc}
\frenchspacing 
\title{Appunti Algoritmi e Strutture Dati}
\author{Luca Seggiani}
\date{13 Marzo 2024}

\usepackage{listings}
\usepackage{xcolor}

\definecolor{codegreen}{rgb}{0,0.6,0}
\definecolor{codegray}{rgb}{0.5,0.5,0.5}
\definecolor{codepurple}{rgb}{0.58,0,0.82}
\definecolor{backcolour}{rgb}{0.95,0.95,0.92}

\lstdefinestyle{code-style}{
    backgroundcolor=\color{backcolour},   
    commentstyle=\color{codegreen},
    keywordstyle=\color{magenta},
    numberstyle=\tiny\color{codegray},
    stringstyle=\color{codepurple},
    basicstyle=\ttfamily\footnotesize,
    breakatwhitespace=false,         
    breaklines=true,                 
    captionpos=b,                    
    keepspaces=true,                 
    numbers=left,                    
    numbersep=5pt,                  
    showspaces=false,                
    showstringspaces=false,
    showtabs=false,                  
    tabsize=2
}

\lstset{style=code-style}

\begin{document}
\maketitle
\section{Algoritmi di teoria dei numeri}
Studiamo adesso una categoria di algoritmi la cui complessità è calcolata prendendo come misura il numero di cifre che compongono il numero. Ad esempio:
\begin{itemize}
  \item L'addizione ha complessità $O(n)$;
  \item La moltiplicazione ha complessità $O(n^2)$.

\end{itemize}
Vediamo ad esempio:
\par\smallskip
\textbf{Moltiplicazione veloce fra interi non negativi} \\
Ogni numero $A$ di $n$ cifre può essere visto come:
$$ A = A_s10^{\frac{n}{2}} + A_d $$
A questo punto il prodotto $AB$ sarà:
$$ AB= A_sB_s 10^{\frac{n}{2}} + A_dB_d $$
che può essere riscritto come:
$$AB = A_sB_S 10^n + ((A_s+A_d)(B_s + B_d) - A_sB_s - A_dB_d)10^{\frac{n}{2}}+A_dB_d $$
dove compaiono tre moltiplicazioni fra interi non con $n$ ma con $\frac{n}{2}$ cifre.
In codice, questo si tradurrà in:
\begin{lstlisting}[language=C++]
  int mult(int A, int B, int n) {
    if(n = 1) return A * B;
    else {
      As = parte sinistra di A; Ad = parte destra di A;
      Bs = parte sinistra di B, Bd = parte destra di B;
      int x_1 = As + Ad; int x_2 = Bs + Bd;
      int y_1 = mult(x_1, x_2, n / 2);
      int y_2 = mult(A, B, n / 2);
      int y_3 = mult(Ad, Bd, n / 2);
      int s_1 = left_shift(y_2, n);
      int s_2 = left_shift(t_1 - t_2- t_3, n / 2);
      return s_1 + s_2 + y_3;
    }
  }
\end{lstlisting}
Dove l'operazione left\_shift(h, n) scorre h di n posti a sinstra, introducento n 0 ($\times 10^n$). Abbiamo tre
chiamate ricorsive alla funzione mult, ciascuna su una dimensione di istanza di $\frac{n}{2}$. Da qui la relazione di istanza:
$$ T(1) = d, \quad T(n) = hn + 3T(\frac{n}{2}) $$
Da cui otteniamo che $ T \in O(n^{\log_2{3}}) = O(n^{1.59})$.
\par\smallskip
\textbf{Relazioni lineari} \\
Relazioni lineari in forma:
$$ T(0) = d, \quad T(n) = bn^k + a_1T(n-1)+a_2T(n-2)+a_rT(n-r) $$
sono polinomiali soltanto se esiste al più una sola chiamata ricorsiva ($a_i = 1$ e $a_j = 0$ con $j \not= i$).
Applichiamo questo teorema al problema del calcolo della serie di Fibonacci. Ricordiamo la classica formulazione ricorsiva:
$$ f_0 = 0, \quad f_1 = 1, \quad f_n = f_{n-1}+f_{n-2} \quad (0, 1, 3, 5, 8, 13...) $$
da cui si ricava:
$$\phi = \lim_{n\rightarrow +\infty} \frac{f_n}{f_{n-1}} = 1.618... $$
nota come sezione aurea (o rapporto aureo). Cerchiamo allora di implementare un algoritmo ricorsivo che restituisca n
cifre della serie di Fibonacci, ad esempio come:
\begin{lstlisting}[language=C++]
int fibonacci(int n) {
  if(n < 2) return n;
  return fibonacci(n - 1) + fibonacci(n - 2);
}
\end{lstlisting}
La relazione di ricorrenza di questo algoritmo è:
$$ T(0) = T(1) = d, \quad T(n) = b + T(n-1) + T(n-2) $$
da cui chiaramente avremo complessità esponenziale (chiaramente poco). Cerchiamo allora di definire un algoritmo iterativo:
\begin{lstlisting}[language=C++]
int fibonacci(int n) {
  int k; int j = 0; int f = 1;
  for(int i = 1; i <= n; ++i) {
    k = j; j = f; f = k + j;
  }
  return j;
}
\end{lstlisting}
visto che il nostro numero di passaggi $g(n)$ è $n$, avremo una complessità totale di $O(n)$. Questo però
significa soltanto che la nostra implementazione ricorsiva di partenza era inefficiente. Troviamone un altra:
\begin{lstlisting}[language=C++]
int fibonacci(int n; int a = 0; int b = 1) {
  if(n == 0)  return a;
  return fibonacci(n - 1, b, a + b);
}
\end{lstlisting}
adesso abbiamo una singola chaiamata ricorsiva su una dimensione di istanza di $(n-1)$, da cui ricaviamo complessità $O(n)$.
\section{Mergesort}
Il mergesort è un algoritmo di ordinamento basato sull'unione, cioè sulla suddivisione del vettore in sottoinsiemi di dimensione minore,
l'ordinamento di quei sottoinsiemi e la loro successiva riunificazione in un unico vettore. In pseudocodice:
\begin{lstlisting}[language=C++]
void mergeSort(sequenza S_1) {
if(S.size() <= 1) return;
else {
  //dividi S in 2 sottosequenze S_1 e S_2 di uguale lunghezza
  sequenza S_2;
  split(S_1, S_2);
  
  mergeSort(S_1);
  mergeSort(s_2);
  
  //fondi S_1 e S_2
  merge(S_1, S_2);
  }
}
\end{lstlisting}
proviamo un'implementazione basata sulle liste:
\begin{lstlisting}[language=C++]
void mergeSort(elem*& s_1) {
  if(s_1 == NULL || s_1->next == NULL) return;
  elem* s_2 = NULL;
  split(s_1, s_2);
  mergeSort(s_1);
  mergeSort(s_2);
  merge(s_1, s_2);
}
\end{lstlisting}
sarà necessario, dalla ricorrenza:
$$ T(n) = bn + 2T(\frac{n}{2}) $$
che una singola chiamata ricorsiva non sia peggiore di $O(n\log{n})$. Vediamo le implementazioni
della split e della merge:
\begin{lstlisting}[language=C++]
void split(elem*& s_1, elem*& s_2) {
  if(s_1 == NULL || s_1->next = NULL) return;
  elem* p = s_1->next;
  s_1->next = p->next;
  p->next = s_2;
  s_2 = p;
  split(s_1->next, s_2);
}
\end{lstlisting}
La funzione split() divide la lista in due "sottoliste", estraendo alternativamente un elemento (da mettere
in una nuova lista) e lasciandone un altro nella lista di partenza.
\begin{lstlisting}[language=C++]
void merge(elem*$ s_1, elem*& s_2) {
  if(s_2 == NULL) return;
  if(s_1 == NULL) {
    s_1 = s_2;
    return;
  }
  if(s_1->inf <= s_2->inf)
    merge(s_1->next, s_2);
  else {
    merge(s_2->next, s_1);
    s_1 = s_2;
  }
}
\end{lstlisting}
La funzione merge() invece scorre le due liste contemporaneamente confrontando di volta in volta i due elementi
concorrenti in testa ad entrambe e scegliendo il minore.
\end{document}
