\documentclass[a4paper,12pt]{article}

\usepackage[french,italian]{babel}
\usepackage[T1]{fontenc}
\usepackage[utf8]{inputenc}
\frenchspacing 
\title{Appunti Algoritmi e Strutture Dati}
\author{Luca Seggiani}
\date{8 Maggio 2024}

\usepackage{listings}
\usepackage{xcolor}

\definecolor{codegreen}{rgb}{0,0.6,0}
\definecolor{codegray}{rgb}{0.5,0.5,0.5}
\definecolor{codepurple}{rgb}{0.58,0,0.82}
\definecolor{backcolour}{rgb}{0.95,0.95,0.92}

\lstdefinestyle{code-style}{
    backgroundcolor=\color{backcolour},   
    commentstyle=\color{codegreen},
    keywordstyle=\color{magenta},
    numberstyle=\tiny\color{codegray},
    stringstyle=\color{codepurple},
    basicstyle=\ttfamily\footnotesize,
    breakatwhitespace=false,         
    breaklines=true,                 
    captionpos=b,                    
    keepspaces=true,                 
    numbers=left,                    
    numbersep=5pt,                  
    showspaces=false,                
    showstringspaces=false,
    showtabs=false,                  
    tabsize=2
}

\lstset{style=code-style}

\begin{document}
\maketitle
Valutiamo la complessità dell'algoritmo di Dijkstra implementato con un min-heap. Abbiamo che il ciclo
while effettua $n$ iterazioni, e che ogni iterazione costa: $$C=C[L_7] + \frac{m}{n}C[L_{15}]$$ ovvero $O(\log{n} + \frac{m}{n}\log{n})$.
La complessità finale sarà quindi: $$O(n(\log{n} + \frac{m}{n}\log{n})) = O(n\log{n} + m\log{n})$$
complessità lineare logaritmica (linearitmica).
\par\smallskip
\textbf{Applicazioni di algoritmi sui grafi} \\
Vediamo alcuni esempi di applicazione dei grafi e degli algoritmi su di essi definiti.
\begin{itemize}
  \item \textbf{Graph coloring} \\
    Si ha un problema di etichettamento dei grafi: dato un numero $K$ di colori da associare ad ogni nodo del grafo, si trovi una colorazione che assicura
    che nessun nodo abbia lo stesso colore di un suo adiacente. Un'esempio di problemi di questo tipo può essere semplicemente il gioco del Sudoku.
  \item \textbf{Algoritmo PageRank di Google} \\
    L'algoritmo PageRank di Google serve al motore di ricerca per trovare le pagine web di interesse per una interrogazione. Cerca di stimare il comportamento
    di un "\textit{random surfer}" che naviga il web basandosi sulle connessioni (\textit{link}) fra le pagine. Riesce a fare questo considerando la rete
    come un grafo (\textit{webgraph}) in cui le pagine sono i nodi, e i link gli archi. Su ogni nodo (quindi per ogni pagina web) $P$ viene calcolato un "rango", che ne rappresenta
    la rilevanza, $R(P)$. $R(P)$ dipende da quanti link arrivano a e provengono da $P$:
    $$ R(P) = \sum_{Q\rightarrow P} \frac{R(Q)}{|Q|} $$
    dove $|Q|$ è il numero di link uscenti da $Q$, e $Q\rightarrow P$ i link da $Q$ a $P$. Questo calcolo viene fatto ottenendo iterativamente il rango dei nodi utilizzando
    la matrice di adiacenza e partendo da un valore del rango uguale per tutti i nodi.
  \item \textbf{Graph database} \\
    I database basati sui grafi forniscono un'alternativa al modello relazionale, sopratutto nei casi in cui le tabelle siano troppe e troppo grandi. Secondo
    questo approccio, i vertici rappresentano tabelle e gli archi relazioni.
\end{itemize}
\end{document}
