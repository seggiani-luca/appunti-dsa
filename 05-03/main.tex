\documentclass[a4paper,12pt]{article}

\usepackage[french,italian]{babel}
\usepackage[T1]{fontenc}
\usepackage[utf8]{inputenc}
\frenchspacing 
\title{Appunti Algoritmi e Strutture Dati}
\author{Luca Seggiani}
\date{3 Marzo 2024}

\usepackage{listings}
\usepackage{xcolor}

\definecolor{codegreen}{rgb}{0,0.6,0}
\definecolor{codegray}{rgb}{0.5,0.5,0.5}
\definecolor{codepurple}{rgb}{0.58,0,0.82}
\definecolor{backcolour}{rgb}{0.95,0.95,0.92}

\lstdefinestyle{code-style}{
    backgroundcolor=\color{backcolour},   
    commentstyle=\color{codegreen},
    keywordstyle=\color{magenta},
    numberstyle=\tiny\color{codegray},
    stringstyle=\color{codepurple},
    basicstyle=\ttfamily\footnotesize,
    breakatwhitespace=false,         
    breaklines=true,                 
    captionpos=b,                    
    keepspaces=true,                 
    numbers=left,                    
    numbersep=5pt,                  
    showspaces=false,                
    showstringspaces=false,
    showtabs=false,                  
    tabsize=2
}

\lstset{style=code-style}

\begin{document}
\maketitle
\section{Introduzione ai programmi ricorsivi}
Iniziamo con l'esempio classico del calcolo del fattoriale. Fissato $0! = 1$, la definizione iterativa è:
$$ n! = 1 \times 2 \times 3 \times ... \times n \quad n>0 $$
un'altra possibilite definizione, forse più naturale, è quella ricorsiva:
$$ n! = n*(n-1)!, \quad n> 0 $$
Vediamo i due approcci in pseudocodice:
\begin{lstlisting}[language=C++]
  int fact(int n) {
    if(n == 0)
      return l; 
  int a = 1;
  for(int i = 1; i <= n; i++) 
    a = a*i;
  return a;
}
\end{lstlisting}

\begin{lstlisting}[language=C++]
int factRic(int n) {
  if(n == 0)
    return 1;
  return n * factRic(n-1);
}
\end{lstlisting}

possiamo dare definizioni ricorsive anche di algoritmi più semplici, come ad esempio la moltiplicazione:
\begin{lstlisting}[language=C++]
int mult(int x, int y) {
int(x == 0)
  return 0;
  return y + mult(x - 1, y);
}
\end{lstlisting}

proviamo a fare l'unrolling ("srotolamento") di questo algoritmo: poniamo una condizione di partenza mult(3,4).
\begin{lstlisting}[language=C++]
mult(3,4)
  4 + mult(2, 4)
    4 + mult(1, 4)
      4 + mult(0, 4) -- fine ricorsione
-> 3 * 4 = 12 
\end{lstlisting}

prendiamo l'algoritmo per il calcolo di un numero pari:
\begin{lstlisting}[language=C++]
int pari(int x) {
  if(x == 0) return 1;
  if(x == 1) return 2;
  return pari(x-2);
}
\end{lstlisting}
oppure l'algoritmo di Euclide nella sua formulazione ricorsiva:
\begin{lstlisting}[language=C++]
int MCD(int x, int y) {
  if(x == y) return x;
  if(x < y) return MCD(x, y - x);
  return MCD(x - y, y);
}
\end{lstlisting}

chiararamente, tutti gli esempi precedentemente riportati rispettano una serie di regole:

\begin{itemize}
  \item Regola 1:
    individuare il caso base in cui la funzione è immediatamente definita
  \item Regola 2:
    effettuare le chiamate ricorsive su un sottinsieme strettamente minore dei dati
  \item Regola 3:
    assicurarsi che la serie di chiamate ricorsive decade sempre su un caso base (in modo da terminare
    la ricorsione, che in caso contrario proseguirebbe all'infinito).
\end{itemize}

vediamo un'esempio di funzioni errate, cioè che non rispettano queste regole:
\begin{lstlisting}[language=C++]
int pari_errata(int x) {
  if(x == 0) return 1;
  //fa che non sia dispari! regole 1 e 2 violate!
  return pari_errata(x - 2);
}
\end{lstlisting}
\begin{lstlisting}[language=C++]
int MCD_errata(int x, int y) {
  if(x == y) return x;
  if(x < y) return MCD_errata(x, y-x);
  return MCD_errata(x, y); 
  //nessuna restrizione del dominio: regola 3 violata!
}
\end{lstlisting}

\section{Programmi ricorsivi su liste}
Diamo una definizione di lista:
\begin{itemize}
  \item Una sequenza vuota è una lista
  \item un elemento seguito da una lista è una lista
\end{itemize}

diamo una definizione della struttura Elem*, nodo della mia lista, in pseudocodice C++:
\begin{lstlisting}[language=C++]
struct Elem {
  int inf;
  Elem* next;
}
\end{lstlisting}

vediamo un algoritmo per calcolare la lunghezza di una lista:
\begin{lstlisting}[language=C++]
int length(Elem* p) {
  if(p == NULL) return 0;
  return 1 + length(p->next);
}
\end{lstlisting}

oppure un'algoritmo che restituisce il numero di volte che un nodo x compare in una lista:
\begin{lstlisting}[language=C++]
int howMany(Elem* p, int x) {
  if(p == NULL) return 0;
  return (p->inf == x) + howMany(p->next, x);
}
\end{lstlisting}

altri algoritmi definiti sulle liste possono essere:
\begin{lstlisting}[language=C++]
int belongs(Elem* l, int x) {
  if(l == NULL) return 0;
  if(l->inf == x) return 1;
  return belongs(l->next, x);
}
\end{lstlisting}

\begin{lstlisting}[language=C++]
void tailDelete(Elem*& l) { //passaggio per riferimento
  if(l == NULL) return;
  if(l-> next == NULL) {
    delete l;
    l = NULL;
  }
  else tailDelete(l->next);
}
\end{lstlisting}

\begin{lstlisting}[language=C++]
void tailInsert(Elem*& l, int x) {
  if(l == NULL) {
    l = new Elem;
    l->inf = x;
    l->next = NULL;
  }
  else tailInsert(l->next, x);
}
\end{lstlisting}

\begin{lstlisting}[language=C++]
void append_1(Elem*& l1, Elem* l2) {
  if(l1 == NULL) l1 = l2;
  else append(l1->next, l2);
}
\end{lstlisting}

\begin{lstlisting}[language=C++]
Elem* append_2(Elem* l1, Elem* l2) {
  if(l1 == NULL)
    return l2;
  l1->next = append(l1->next, l2);
  return l1;
}
\end{lstlisting}

\section{Induzione}
Riportiamo brevemente la definizione di induzione sui numeri naturali:
\begin{center}
Sia una certa proprietà $P$ vera per un naturale $n_0$. Se è vero che $P(n) \Rightarrow P(n + 1)$, allora $P$ 
è vera per qualsiasi naturale $n > n_0$.
\end{center}

L'induzione può essere usata ad esempio per dimostrare che la somma dei primi n naturali è:
$$ \sum_{i = 1}^{n} i = \frac{n(n+1)}{2} $$  
introduciamo anche l'induzione completa:
\begin{center}
Se $P$ vale per ogni naturale $m <= n$, e per ogni $n$ vale per $n+1$, allora $P$ vale per ogni naturale
\end{center}
e l'induzione ben fondata:
\begin{center}
Sia $S$ un certo insieme ben ordinato. Se la proprietà $P$ vale per i minimali di $S$, e per ogni elemento $E \in S$
$P$ è vera su $E$ se è vera su ogni elemento minore di $E$, allora $P$ vale su tutto l'insieme $S$.

\footnotesize
(si noti che il totale abuso di notazione $P(x)$ indica $P$ vera su $x$).
\normalsize

\end{center}

\section{Complessità dei programmi ricorsivi}
Facciamo il quadro della situazione, applicando quello che abbiamo visto finora e quello che avevamo già definito
sugli algoritmi iterativi. Consideriamo il seguente algoritmo per il calcolo di $n!$:
\begin{lstlisting}[language=C++]
int fact(int x) {
  if(x == 0) return l;
  else return x * fact(x - 1);
}
\end{lstlisting}
l'algoritmo non fa altro che definire la relazione di ricorrenza:
$$ T(0) = a, \quad T(n) = b + T(n - 1) $$
andando a svolgere la serie di operazioni definite dall'algoritmo avremo:
$$ T(0) = a $$
$$ T(1) = b + a $$
$$ T(2) = b + b + a = 2b + a $$
$$ T(3) = b + 2b + a = 3b + a $$
$$ ... $$
$$ T(n) = nb + a $$
l'ultima funzione, $T(n)$, non dipende da alcun passo ricorsivo e fornisce un'interpretazione della complessità
dell'algoritmo (che è O(n)).
\par\smallskip
\textbf{Selection sort} \\
Applichiamo quanto visto finora ad un programma più complicato: il selection sort, formulato ricorsivamente.
\begin{lstlisting}[language=C++]
void r_selectionSort(int* A, int m, int i = 0) {
  if(i == m - 1) return;
  int min = i;
  for(int j = i + 1; j < m; j++)
    if(A[j] < A[min]) min = j;
  exchange(A[i], A[min]);
  r_selectionSort(A, m, i + 1);
}
\end{lstlisting}

la relazione di ricorrenza in questo caso sarà:
$$ T(1) = a, \quad T(n) = bn + T(n - 1) $$ \newpage
che potremo svolgere come:
$$ T(1) = a $$
$$ T(2) = 2b + a $$
$$ T(3) = 3b + 2b + a $$
$$...$$
$$ T(n) = (n + n-1 + n-2 + ... + 2)b + a = (\frac{n(n+1)}{2} - 1) b + a $$

\section{Introduzione al quicksort}
L'algoritmo quicksort è un'algoritmo di ordinamento particolarmente efficiente, schematicamente così definito:
\begin{itemize}
  \item Scelta di un perno
  \item Divisione dell'array in 2 sottoinsiemi: elementi minori del perno ed elementi maggiori del perno
  \item Esegui la stessa operazione, ricorsivamente, sui 2 sottoinsiemi finchè non ne incontri uno di 2 elementi.
\end{itemize}

nel dettaglio, la divisione in due sottoinsiemi dell'array avviene mediante scambi fra elementi utilizzando due cursori
(siano s e d), inizialmente posti, rispettivamente, ad inf e sup. Da qui in poi:
\begin{itemize}
  \item Finché s < d, \\
    porta avanti s finche A[s] è minore del perno; \\
    porta indietro d fino a che A[d] è maggiore del perno; \\
    scambia A[s] e A[d];
    incrementa S e decrementa D.
  \item Se inf < d ripeti il quicksort sulla parte da inf a d.
  \item Se s < sup ripeti il quickosrt sulla parte da s a sup.
\end{itemize}

Vediamo infine lo pseudocodice:
\begin{lstlisting}[language=C++]
void quickSort(int A[], int inf = 0, int sup = n -1) {
  int perno = A[(inf + sup) / 2]; //prendo l'elemento centrale come perno
  int s = inf, d = sup;
  while(s < d) {
    while(A[s] < perno) s++;
    while{A[d] > perno} d--;
    if(s > d) break;
    exchange(A[s], A[d]);
    s++;
    d--;
  }

  if(inf < d)
    quickSort(A, inf, d);
  if(s < sup)
    quickSort(A, s, sup);
}
\end{lstlisting}

\end{document}
