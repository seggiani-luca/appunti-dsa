\documentclass[a4paper,12pt]{article}

\usepackage[french,italian]{babel}
\usepackage[T1]{fontenc}
\usepackage[utf8]{inputenc}
\frenchspacing 
\title{Appunti Algoritmi e Strutture Dati}
\author{Luca Seggiani}
\date{9 Aprile 2024}

\usepackage{listings}
\usepackage{xcolor}

\definecolor{codegreen}{rgb}{0,0.6,0}
\definecolor{codegray}{rgb}{0.5,0.5,0.5}
\definecolor{codepurple}{rgb}{0.58,0,0.82}
\definecolor{backcolour}{rgb}{0.95,0.95,0.92}

\lstdefinestyle{code-style}{
    backgroundcolor=\color{backcolour},   
    commentstyle=\color{codegreen},
    keywordstyle=\color{magenta},
    numberstyle=\tiny\color{codegray},
    stringstyle=\color{codepurple},
    basicstyle=\ttfamily\footnotesize,
    breakatwhitespace=false,         
    breaklines=true,                 
    captionpos=b,                    
    keepspaces=true,                 
    numbers=left,                    
    numbersep=5pt,                  
    showspaces=false,                
    showstringspaces=false,
    showtabs=false,                  
    tabsize=2
}

\lstset{style=code-style}

\begin{document}
\maketitle
\section{Alberi binari di ricerca, implementazione}
Vediamo l'implementazione effettiva di un albero binario di ricerca, effettuata in C++ attraverso il meccanismo delle classi:
\begin{lstlisting}[language=C++]
struct Node {
  int value;
  Node* left;
  Node* right;

  Node(int val):
    value(val), left(NULL), right(NULL) {}
};

class BinTree{
  Node root_;

public:
  BinTree() { root_ = NULL; }
  Node* getRoot() { return root_; }
};
\end{lstlisting}

La insert:
\begin{lstlisting}[language=C++]
void insert(int val) {
  Node* node = new Node(val);
  Node* pre = NULL;
  Node* post = root_;

  while(post != NULL) {
    pre = post;
    if(val <= post->value)
      post = post->left;
    else
      post = post->right;
  }
  
  if(pre == nullptr) {
    root_ = node;
    return;
  }

  if(val <= pre->val)
    pre->right = val;
  else
    pre->left = val;

}
\end{lstlisting}

Funzioni per la ricerca di minimi e massimi:
\begin{lstlisting}[language=C++]
Node* min() {
  Node* temp = root_;
  while(temp->left != NULL)
    temp = temp->left;
  return temp;
}

Node* max() {
  Node* temp = root_;
  while(temp->right != NULL)
    temp = temp->right;
  return temp;
}
\end{lstlisting}

Funzioni per la ricerca di la profondità di minimi e massimi:
\begin{lstlisting}[language=C++]
int height(Node* tree) {
  int hLeft;
  int hRight;
  if(tree == NULL)
    return 0;
  hLeft = height(tree->left);
  hRight = height(tree->right);
  
  return(1 + max(hLeft, hRight));
}
\end{lstlisting}
Funzioni di ricerca di nodi:
\begin{lstlisting}[language=C++]
Node* search(int val) {
  Node* temp = root_;
  while(temp != nullptr) {
    if(val == temp->value)
      return temp;
    if(val <= temp->value)
      search(temp->left);
    else
      search(temp->right);
  }
  return nullptr;
}
\end{lstlisting}
\section{Stringhe STL}
La libreria STL definisce nell'header <string> il tipo di dato stringa. Il tipo di dato stringa estende in qualche
modo la stringa del comune C, ovvero un'array di tipi carattere terminata da un carattere specifico. Attraverso l'incapsulamento
di questo tipo di dati, si possono avere diverse comodità, come ad esempio il conto della lunghezza della lunghezza della stringa
senza necessità di scorrimenti, o funzioni membro come:
\begin{lstlisting}[language=C++]
string stringa = "Il contrario di comunismo";
string stringa_2 = " e' egoismo";
string stringa_completa = stringa + stringa_2; 
  //stringa_completa = "Il contrario di comunismo e' egoismo"
stringa.find("comunismo"); 
  //puntatore alla prima occorrenza di "comunismo", senno' :npos
stringa_2.compare("egoismo");
...
\end{lstlisting}
\end{document}
